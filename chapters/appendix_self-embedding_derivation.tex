
%%%%%%%%%%%%%%%%%%%%%%%%%%%%%%%%%%%%%%%%%%%%%%%%%%%%%%%%%%%%%%%%%%%%%%%%%%%%%%
\section {boundry conditions}
%%%%%%%%%%%%%%%%%%%%%%%%%%%%%%%%%%%%%%%%%%%%%%%%%%%%%%%%%%%%%%%%%%%%%%%%%%%%%%

step 1.) 

Assumptions:

transport mean free path is normalized to unity

L $equiv$ system length

l $equiv$ transport mean free path

k $equiv$ wave number

\begin{equation}
l \ll L \leq N l
\end{equation}

weak scattering approximation: $k l \ll 1$

$L \ll N l \rightarrow$ diffusive regime

$N l \ll L \rightarrow$ localization regime


step 2.) 

In general, the electric field propagating through medium as a function of position for channel n
\begin{equation}
E_n(x) = E_n^+ \exp(i k_{\parallel n} x) + E_n^- \exp(-i k_{\parallel n} x) 
\end{equation}
For the bulk media with light incident on the left for N channels with the input channel $n_0$, $ 1 \leq n_0 \leq N$, the $n^{th}$ channel electric field a position $x$ is described by
\begin{center}
\begin{tabular}{cc}
$n = n_0$ \quad \quad \quad & $ E_{n_0}(x) = 1 \exp(i k_{\parallel n_0} x) + r_{n_0n_0} \exp(-i k_{\parallel n_0} x) $ \\
$n \ne n_0$  \quad \quad \quad & $ E_n(x) = 0 + r_{n_0n} \exp(-i \kappa_{\parallel n} x) $ \\
\end{tabular}
\end{center}
Where the coefficient 1 is the input amplitude. The coefficient $r_{n_0n_0}$ is reflection.  There are two indices, the first refering to the input channel and the second is channel number. The 0 term of the second equation denotes that for channels other than the input channel the input is zero.
% why does the kappa have an "i" in the second equation?  Why kappa?
The derivative of the electric field is described similarly. Again, there is a difference between between the equations for the input channel $n_0$ and all other channels.
\begin{center}
\begin{tabular}{cc}
$n = n_0$ \quad \quad \quad & $ \frac{1}{k_{\parallel n_0}} E_{n_0}^{\prime}(x) = i \exp(i k_{\parallel n_0} x) - i r_{n_0n_0} \exp(-i k_{\parallel n_0} x) $ \\
$n \ne n_0$  \quad \quad \quad & $ \frac{1}{k_{\parallel n_0}} E_n^{\prime}(x) = 0 -i r_{n_0n} \exp(-i \kappa_{\parallel n} x) $ \\
\end{tabular}
\end{center}
The channels other than the input channel has incident value 0.


step 3.)

To set up input boundry conditions on the incident side of the medium, $(x=0)$, the input vector has the form
\begin{equation}
\vec{v}(0) = 
\left( \begin{array}{c}
r_{n_01} \\     % field, open: first channel
r_{n_02} \\     % field, open: second channel
\vdots \\       % ...
1+r_{n_0n_0} \\	% field, open: input channel
\vdots \\       % ...
r_{n_0N} \\     % field, open: last channel
r_{n_0N+1} \\   % field, closed: first channel 
\vdots \\       % ...
r_{n_0n_{max}}\\ % field, closed: last channel
-i r_{n_01} \\  % deriv, open: first channel
-i r_{n_02} \\  % deriv, open: second channel
\vdots \\       % ...
i - i r_{n_0n_0} \\ % deriv, open: input channel
\vdots \\       % ...
-i r_{n_0N} \\  % deriv, open: last channel
-i r_{n_0N+1} \\ % deriv, closed: first channel
\vdots \\       % ...
r_{n_0n_{max}} \\ % deriv, closed: last channel
\end{array} \right)
\label{incidentsidevector}
\end{equation}
Again, the first index keeps track of the input channel, the second index is the channel number. The size of $\vec{v}(0)$ is $N+N_c+N+N_c$, where $n_{max} = N+N_c$.  The upper half is the field, the lower half is the derivative of the field.  Each half is broken into $N$ open channels and $N_c$ closed channels.

For $\vec{v}(0)$ the input channel is $n_0$. Note that when the two indices match, there is a $1+$ which is the input channel incident amplitude.  There is no input to the closed channels as we assume the incident beam is from infinity. Thus no near-field contribution to input.  

The two vectors are related by a matrix that describes the propagation through the medium.
\begin{equation}
\hat{M}(0,L) \vec{v}(0) = \vec{v}(L)
\end{equation}

as long as disorder and $\lambda$ remain the same (?).

step 4.)

input channels

In general, the electric fields for open and closed channels are
\begin{center}
\begin{tabular}{cc}
open \quad \quad \quad & $ E_n(x) = E_n^+ \exp(i k_{\parallel n} x) + E_n^- \exp(-i k_{\parallel n} x) $ \\
closed  \quad \quad \quad & $ E_n(x) = E_n^+ \exp(-\kappa_{\parallel n} x) + E_n^- \exp(\kappa_{\parallel n} x) $ \\
\end{tabular}
\end{center}
where $i \kappa \equiv k$

Note that on the exit side of the medium the second term is zero.

Setting up the transmission side,
\begin{equation}
\vec{v}(L) = 
\left( \begin{array}{c}
t_{n_01} \\      % field, open: first channel
t_{n_02} \\      % field, open: second channel
\vdots \\        % ...
t_{n_0N} \\	 % field, open: last channel
t_{n_0N+1} \\    % field, closed: first channel 
\vdots \\        % ...
t_{n_0n_{max}} \\ % field, closed: last channel
i t_{n_01} \\    % deriv, open: first channel
i t_{n_02} \\    % deriv, open: second channel
\vdots \\        % ...
i t_{n_0N} \\	 % deriv, open: last channel
-t_{n_0N+1} \\   % deriv, closed: first channel
\vdots \\        % ...
-t_{n_0n_{max}} \\ % deriv, closed: last channel
\end{array} \right)
\label{transmissionsidevector}
\end{equation}
Where the vector is similar to \ref{incidentsidevector}.  The size of $\vec{v}(L)$ is $2 n_{max}$ where $n_{max} = N+N_c$ and there are $N$ open channels, $N_c$ closed channels. The upper half is the field coefficients, the lower half is the electric field derivative. Unlike \ref{incidentsidevector} there is no input.

The positive $i$ in the open channels of the derivative (lower half) denotes forward propagation.

The $k$ and $kappa$ are normalized to 1 and do not appear in $\vec{v}(L)$.

%%%%%%%%%%%%%%%%%%%%%%%%%%%%%%%%%%%%%%%%%%%%%%%%%%%%%%%%%%%%%%%%%%%%%%%%%%%%%%
\section {Self-embedding method}
%%%%%%%%%%%%%%%%%%%%%%%%%%%%%%%%%%%%%%%%%%%%%%%%%%%%%%%%%%%%%%%%%%%%%%%%%%%%%%

Overview:

multiplying many transfer matrices results in divergent eigenvalues.  To limit the growth of the divergence, periodic renormalization allows the user to extend how many transfer matrices can be multiplied.

In a simplified example, given a large number of transfer matrices, multiply a first small subset on the incident side. Multiply a second subset and renormalize, then combine the two chunks. Repeat iteratively for as many subsets as make up the system length.

One advantage to this approach is that each chunk is independent of the boundry condition.

%%%%%%%%%%%%%%%%%%%%%%%%%%%%%%%%%%%%%%%%%%%%%%%%%%%%%%%%%%%%%%%%%%%%%%%%%%%%%%
\subsection {Determining matrices}
%%%%%%%%%%%%%%%%%%%%%%%%%%%%%%%%%%%%%%%%%%%%%%%%%%%%%%%%%%%%%%%%%%%%%%%%%%%%%%

Self-embedding allows for boundry condition-independent calculation.  To allow this, re-write boundary conditions in the form 
\begin{equation}
\hat{G}\vec{v}(0)+\hat{H}\vec{v}(L) = \vec{v}_{BC}
\end{equation}
where $\vec{v}_{BC}$ should not contain any unknown variables. $\vec{v}_{BC}$ should not depend on transmission or reflection.  Any $\hat{G}$ and $\hat{H}$ that results in this is valid (not a unique solution). 

One solution is
\begin{equation}
\hat{G} = 
\left( \begin{array}{ccccccc|ccccccc}
  1    &   0   & \ldots & | &        &        &        &    -i  &  0  & \ldots & | &        &    &        \\
  0    &   1   &        & | &        &   0    &        &     0  & -i  &        & | &        &  0 &        \\
\vdots &       & \ddots & | &        &        &        & \vdots &     & \ddots & | &        &    &        \\ 
  -    &   -   &   -    & - &   -    &   -    &   -    &    -   &  -  &   -    & - &   -    & -  &        \\
       &       &        & | &    1   &   0    & \ldots &        &     &        & | &   -1   &  0 & \ldots \\
       &   0   &        & | &    0   &   1    &        &        &     &   0    & | &   0    & -1 &        \\
       &       &        & | & \vdots &        & \ddots &        &     &        & | & \vdots &    & \ddots \\
\hline
       &       &        & | &        &        &        &        &     &        & | &        &    &        \\
       &   0   &        & | &        &   0    &        &        &  0  &        & | &        &  0 &        \\
       &       &        & | &        &        &        &        &     &        & | &        &    &        \\
  -    &   -   &   -    & - &    -   &    -   &   -    &   -    &  -  &  -     & | &   -    & -  &   -    \\
       &       &        & | &        &        &        &        &     &        & | &        &    &        \\
       &   0   &        & | &        &   0    &        &        &  0  &        & | &        &  0 &        \\
       &       &        & | &        &        &        &        &     &        & | &        &    &        \\
\end{array} \right)
\label{Gmatrix}
\end{equation}
The upper left quadrant manipulates the field, the upper right operates on the derivative. Each quadrant is divided into four parts: the upper left is open, the lower left is closed.

Note that multiplying \ref{Gmatrix} by \ref{incidentsidevector} is $1(1+r) + -i(i - ir) = 2$ for the input channel and zero for all other channel values. The gives the upper half of the $\vec{v}_{BC}$ vector

\begin{equation}
\vec{v}_{BC} =
\left( \begin{array}{c}
0 \\
2 \\
0 \\
\vdots \\
0 \\
-- \\
0 \\
\vdots \\
0 \\
\hline
0 \\
\vdots \\
0 \\
-- \\
0 \\
\vdots \\
0 \\
\end{array} \right)
\label{vbc}
\end{equation}
The ``2'' occurs at the input channel, $n=n_0$.  Everything else being zero means there is no dependence on transmission, reflection, or any other variable.

Once $\hat{G}$ is determined,  $\hat{H}$ must have a form that does not introduce variables
(dependencies) into $\vec{v}_{BC}$.
\begin{equation}
\hat{H} =
\left( \begin{array}{ccccccc|ccccccc}
       &       &        & | &        &        &        &        &     &        & | &        &    &        \\
       &   0   &        & | &        &   0    &        &        &  0  &        & | &        &  0 &        \\
       &       &        & | &        &        &        &        &     &        & | &        &    &        \\
  -    &   -   &   -    & - &    -   &    -   &   -    &   -    &  -  &  -     & | &   -    & -  &   -    \\
       &       &        & | &        &        &        &        &     &        & | &        &    &        \\
       &   0   &        & | &        &   0    &        &        &  0  &        & | &        &  0 &        \\
       &       &        & | &        &        &        &        &     &        & | &        &    &        \\
\hline
  i    &   0   & \ldots & | &        &        &        &    -1  &  0  & \ldots & | &        &    &        \\
  0    &   i   &        & | &        &   0    &        &     0  & -1  &        & | &        &  0 &        \\
\vdots &       & \ddots & | &        &        &        & \vdots &     & \ddots & | &        &    &        \\ 
  -    &   -   &   -    & - &   -    &   -    &   -    &    -   &  -  &   -    & - &   -    & -  &        \\
       &       &        & | &    1   &   0    & \ldots &        &     &        & | &    1   &  0 & \ldots \\
       &   0   &        & | &    0   &   1    &        &        &     &   0    & | &   0    &  1 &        \\
       &       &        & | & \vdots &        & \ddots &        &     &        & | & \vdots &    & \ddots \\
       \end{array} \right)
\label{Hmatrix}
\end{equation}

%%%%%%%%%%%%%%%%%%%%%%%%%%%%%%%%%%%%%%%%%%%%%%%%%%%%%%%%%%%%%%%%%%%%%%%%%%%%%%
\subsection {self-embedding implementation}
%%%%%%%%%%%%%%%%%%%%%%%%%%%%%%%%%%%%%%%%%%%%%%%%%%%%%%%%%%%%%%%%%%%%%%%%%%%%%%

The initial self-embedding matrix is 
\begin{equation}
\hat{S}(0,0) = (\hat{G}+\hat{H})^{-1}
\end{equation}

transmission:
\begin{equation}
\hat{S}(J+1,J+1) = \hat{M}_J \hat{\sum}(J) \hat{S}(J,J)
\end{equation}
where
\begin{equation}
\hat{\sum}(J) \equiv \left[ \hat{I} - \hat{S}(J,J) \hat{H} \left( \hat{I} - \hat{M}(J) \right) \right]^{-1}
\end{equation}

reflection:
\begin{equation}
\hat{S}(0,J) = \hat{S}(0,J) \left[ \hat{I} + \hat{H} \left( \hat{I} - \hat{M}(J) \right) \hat{\sum}(J) \hat{S}(J,J) \right]
\end{equation}

\begin{equation}
\begin{gathered}
\vec{v}_N = \hat{S}(N,N) \vec{v}_{BC} \\
\vec{v}_0 = \hat{S}(0,N) \vec{v}_{BC}
\end{gathered}
\end{equation}

then $v_t(1:N_{total},1)$ need to add to $t=(1:N_{total},1:N_{open})$ the last self-embedding step


\begin{equation}
\vec{v}(0) = 
\left( \begin{array}{ccc|ccc}
           &         &    &  &   & \\
           & \hat{I} &    &  & 0 & \\
           &         &    &  &   & \\
\hline
 -i\hat{I} &   |     & 0  &  &   & \\
     -     &   -     & -  &  & 0 & \\
     0     &   |     & 0  &  &   & \\
\end{array} \right)
\left( \begin{array}{c}
\\
 \vec{r} \\
 \\
\hline
  \\
0 \\
  \\
\end{array} \right)
+
\left( \begin{array}{c}
\\
 \vec{\delta}_{n_0} \\
 \\
\hline
  \\
i\vec{\delta}_{n_0} \\
  \\
\end{array} \right)
= \hat{S}_r 
\left( \begin{array}{c}
 \vec{r} \\
 0
\end{array} \right)
+\vec{p}
\end{equation}


\begin{equation}
\vec{v}(L) = 
\left( \begin{array}{ccc|ccc}
  &   &  &         &         &        \\
  & 0 &  &         & \hat{I} &         \\
  &   &  &         &         &         \\
\hline
  &   &  & \hat{I} &   |     &   0     \\
  & 0 &  &   -     &   -     &   -     \\
  &   &  &   0     &   |     & \hat{I} \\
\end{array} \right)
\left( \begin{array}{c}
\\
0 \\
\\
\hline
\\
\vec{t} \\
\\
\end{array} \right)
= \hat{S}_t
\left( \begin{array}{c}
0 \\
\vec{t}
\end{array} \right)
\end{equation}

\begin{equation}
\hat{M}\hat{S}_r
\left( \begin{array}{c}
\vec{r}
0 \\
\end{array} \right)
+\hat{M}\vec{p} = \hat{S}_t 
\left( \begin{array}{c}
0 \\
\vec{t}
\end{array} \right)
\end{equation}

\begin{equation}
\hat{M}\hat{S}_r
\left( \begin{array}{c}
\vec{r}
0 \\
\end{array} \right)
-\hat{S}_t 
\left( \begin{array}{c}
0 \\
\vec{t}
\end{array} \right) =
-\hat{M}\vec{p}
\end{equation}

\begin{equation}
\hat{c}
\left( \begin{array}{c}
\vec{r} \\
\vec{t}
\end{array} \right)
= -\hat{M}\vec{p}
\end{equation}

Once $\hat{S}(N,N)$ is known for the medium, the transmission is extracted one input channel at a time.

\begin{equation}
\hat{S}(N,N) \vec{v}_{BC} = \vec{t}
\end{equation}

For each input channel, there is an output transmission vector. To construct the transmission matrix, which relates input channel to output channel, join the transmission outputs.

\begin{equation}
\hat{T} = [ \vec{t} \vec{t} ... \vec{t} ]
\label{transmissionMatrix}
\end{equation}
