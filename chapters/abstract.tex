%\specialhead{Abstract}
%\textbf{Abstract}
%\addcontentsline{toc}{lof}{Abstra}
\begin{spacing}{1.55}
Passive quasi-one-dimensional random media exhibit one of the three regimes of transport -- ballistic, diffusive, or Anderson localization -- depending on system size. The ballistic and diffusion approximations assumes particle transport, whereas Anderson Localization occurs when wave self-interference effects are dominant. When the system contains absorption or gain, then how the regimes can be characterized becomes unclear. By investigating theoretically and numerically the ratio of transmission to energy in a random medium in one dimension, we show this parameter can be used to characterize localization in random media with gain. 

Non-conservative media implies a second dimension for the transport parameter space, namely gain/absorption. By studying the relations between the transport mean free path, the localization length, and the gain or absorption lengths, we enumerate fifteen regimes of wave propagation through quasi-one-dimensional non-conservative random media. Next a criterion characterizing the transition from diffusion to Anderson localization is developed for random media with gain or absorption. The position-dependent diffusion coefficient, which is closely related to the ratio of transmission to energy stored in the system, is investigated using numerical models.

In contrast to random structures, deterministic aperiodic structures (DAS) offer predictable and reproducible transport behaviors while exhibiting a variety of unusual transport properties not found in either ordered or random media. By manipulating structural correlations one may design and fabricate artificial photonic nanomaterials with prescribed transport properties. 
%In addition to numerical studies of random media, we can use the same tools to investigate correlated disorder. 
%Deterministic aperiodic structures, such as that obtained via the Thue-Morse algorithm,  
The Thue-Morse structure is a prime example of deterministic aperiodic systems with singular-continuous spatial Fourier spectra. 
The non-periodic nature of the system makes it notoriously difficult to characterize theoretically especially in dimensions higher than one.  The possibility of mapping the two-dimensional aperiodic Thue-Morse pattern of micro-cavities onto a square lattice is demonstrated, making it amenable to the tight-binding description. 
%Once coupling coefficients are found, we can apply these in a Hamiltonian in the tight-binding model. We study the optical properties of a two-dimensional array of micro-cavities spatially arranged according to the Thue-Morse sequence. %In a realistic system we establish applicability of the tight-binding description. It is employed to i
%We investigate coexisting localized and delocalized states and their scaling dependence on the size of the structure.

\end{spacing}

%Abstract as breakdown of title ``Criterion for Anderson Localization in Active Random Media''
%\begin{itemize}
%\item study = numerical, analytical, theory
%\item Anderson Localization = diffusion with interference effects for passive systems
%\item active = not passive: gain or absorption
%\item random media = randomly placed delta function ``scatterers'' 
%\end{itemize}