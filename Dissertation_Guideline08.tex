%&LaTeX
\documentclass{article}
\usepackage{longtable}
\usepackage{multirow}
\usepackage{amsmath}                                                                           
                                                                                                    
\newcommand{\tab}{\hspace{5mm}}

\begin{document}


\newpage
\begin{center}
THE TITLE OF DISSERTATION

GOES HERE -- DOUBLE

SPACED


\end{center}


\begin{center}
by

YOUR FULL NAME HERE


\end{center}


\begin{center}
A DISSERTATION

Presented to the Faculty of the PlaceNameplaceGraduate PlaceTypeSchool 
of the 

placePlaceNameMISSOURI PlaceTypeUNIVERSITY OF SCIENCE \& TECHNOLOGY

In Partial Fulfillment of the Requirements for the Degree


\end{center}

\begin{center}
DOCTOR OF PHILOSOPHY\\
in\\
MECHANICAL ENGINEERING


\end{center}


\begin{center}
2008

Approved by

Robert Smith, Advisor\\
James B. Brown\\
Charles R. Jones\\
D. F. Ling\\
Hardy Davidson


\end{center}


Students do not own the copyrights of research which is sponsored 
by grants or by stipend from UMR. Check with your advisor if 
you are unsure. You need to leave this page blank in that case.


\begin{center}
{\copyright} 2008\\
Your Full Name\\
All Rights Reserved\\

\end{center}

14






First page with visible page number.

ABSTRACT


Abstract is supposed to be a short description of your work. 
No references, equations, or Figures go here. It should be one 
page maximum (250 to 350 words). It must be double spaced since 
the library puts it on microfilm.

\newpage

\begin{center}
ACKNOWLEDGMENTS


\end{center}


Should not exceed one page.

This is where you should thank your Advisor for teaching you 
how to walk on water. Further, you should acknowledge any and 
all funding sources. \\
For professional courtesy and for brownie points with your committee, 
you should thank the members of your committee. 


Finally, thanking family members or someone important to you 
is okay as well.


\newpage

\begin{center}
TABLE OF CONTENTS


\end{center}

Refer to the thesis/dissertation specifications for illustration 
of ``Table of Contents''.

Page


ABSTRACT\tab i\\
ACKNOWLEDGMENTS\tab i\\
LIST OF ILLUSTRATIONS\tab i\\
LIST OF TABLES\tab i\\
NOMENCLATURE\tab i


SECTION


1. INTRODUCTION\tab 1


1.1. THIS IS A SUBSECTION\tab 1


1.1.1. How to Use the STYLE Formats\tab 1\\
1.1.2. How to Use the Table of Contents\tab 1\\
1.1.3. Formatting Figures\tab 1\\
1.1.4. Formatting Tables\tab 1


1.2. MORE EQUATIONS AND FIGURES\tab 1


2. MISCELLANEOUS INFORMATION\tab 1


2.1. THIS IS SUBSECTION LEVEL ONE\tab 1\\
2.2. THIS IS SUBSECTION LEVEL TWO\tab 1


2.2.1. Second Subsection\tab 1\\
2.2.2. Second Subsection\tab 1


2.2.2.1 Third subsection\tab 1\\
2.2.2.2 Third subsection\tab 1


2.2.2.2.1 Fourth subsection\tab 1


2.2.2.2.2 Fourth subsection will illustrate the case with very 


\tab long title\dots \tab 1


3. THIS IS SECTION THREE\tab 1


3.1. THIS IS SUBSECTION LEVEL ONE\tab 1


3.1.1. Second Subsection\tab 1\\
3.1.2. Second Subsection\tab 1


3.1.2.1 Third subsection\tab 1\\
3.1.2.2 Third subsection\tab 1


3.1.2.2.1 Fourth subsection\tab 1\\
3.1.2.2.2 Fourth subsection\tab 1


APPENDICES\tab 9


A. THIS IS AN APPENDIX\tab 9\\
B. THIS IS AN APPENDIX\tab 11


BIBLIOGRAPHY\tab 1\\
VITA\tab \tab 1



Refer to the thesis specifications for examples.




\newpage

\begin{center}
LIST OF ILLUSTRATIONS


\end{center}

Figure\tab \tab \tab \tab \tab \tab \tab \tab \tab \tab \tab  Page


1.1. Example Four-Bar Linkage\tab 1\\
1.2. Four-Bar Linkage -- Crossed Configuration\tab 1\\
1.3. Example Mechanism\tab 1\\
2.1. Vector Loop Decomposition of Loop 1\tab 1



\newpage

\begin{center}
LIST OF TABLES


\end{center}

Table\tab \tab \tab \tab \tab \tab \tab \tab \tab \tab \tab  Page


1.1. Original STATUS Array for Example Mechanism\tab 1



\newpage

\begin{center}
NOMENCLATURE


\end{center}


This section is optional.

Symbol\tab Description\tab \tab \tab \tab \tab \tab \tab \tab 


\ensuremath{\beta}\tab  \tab  Angle of Attack\\
\ensuremath{\alpha}\tab \tab  Change in Coupler Angle








\section*{1.\tab }


\section*{

}

It is a good idea to retain Body Text as unjustified (Align Left). 
You'll have fewer problems while printing this way.

INTRODUCTION\\


\subsection*{1.1.\tab }


\subsection*{
THIS IS A SUBSECTION}

This is where the dissertation really begins. The most important 
thing to remember is that first person (I, We, our, my, etc.) 
should not to be used in formal writings. This is a sentence 
to take up space and look like text. In fact, this text and ALL 
text within the body of the T/D uses the STYLE Body Text, usually 
found on the toolbar next to the font type or, alternatively, 
in the Format menu under Style\dots . All headings, text, Figure 
titles, and table titles have been mapped to the STYLE formats 
and should be used throughout your T/D in order for the Table 
of Contents, Figures, \dots etc to function correctly. \\


\subsubsection*{1.1.1.\tab }


\subsubsection*{
How to Use the STYLE Formats. Note that this is a second level}

subsection heading and is indented \ensuremath{\frac12}'' over from the 
left hand margin, and it's text is NOT all capital letters while 
the paragraph text begins immediately after the heading. Also 
note that because of the way Word handles styles, this subsection 
heading has to be doctored a little. The entire first line is 
Heading 3 style, but only the title is bold. How do you do that? 
Go ahead and put 2 spaces after the period, and begin typing 
the first line of your body. Then, once you get to the second 
line, highlight from the first space following the period to 
the end of the first line (level above), and turn off the bold 
and underline properties. Once those are off, you then have to 
give a carriage return, and you should then be automatically 
placed into the Body Text style. Note that you must be careful 
when using these carriage returns and be prepared to doctor it 
up. ``But wait a minute, the next line becomes tabbed over!'' 
To get rid of this, just backspace it out. The STYLE to use for 
headings and titles are as follows: 


\tab all headings in the pre-text pages (i -- x) that you want in the 
Table of Contents 


should be Heading 0


\tab the only heading you wouldn't want in the table of contents, 
the TABLE OF 


CONTENTS,\tab should be the style TOC HEADING


all major Section headings, such as INTRODUCTION above, are Heading 
1


\tab all subsection headings follow in a like manner, Heading 2 (see 
1.1 THIS IS A 


SUBSECTION above), Heading 3, \dots etc.


\tab all body text, except for the Abstract, should be style Body 
Text\\
\tab the abstract body text should be the style Abstract Body. It 
makes the paragraph\\
\tab \tab double spaced automatically.\\
\tab all Figure captions should be the style Figure Title\\
\tab all table titles should be Table Title\\


\subsubsection*{1.1.2.\tab }


\subsubsection*{
How to Use the Table of Contents. The TABLE OF CONTENTS, LIST}

OF FIGURES, LIST OF TABLES, and NOMENCLATURE all work in a similar manner. 
The table of contents (TOC) is automatically generated and uses 
the styles mentioned above. They are already linked together, 
so once you get your T/D up and running, all you have to do is 
right click in the TOC and click Update Fields (and Update Entire 
Table if prompted) and Word then searches out the styles Heading 
0, 1, 2, \dots etc. However, note that for Sections that use Heading 
3 and beyond, you may have to modify the entry in the TOC to 
exclude the rest of the first line. 


The LIST OF FIGURES (LOF) and LIST OF TABLES (LOT) work in a 
similar manner, except they use the styles Figure Title and Table 
Title, respectively. Again, to update these, you will have to 
right click the table and click Update Field. Also, because the 
title of Figures and tables usually begin with Figure and Table, 
you will again have to update the entry in the table to remove 
these words to avoid the repetition. If you leave the word Figure 
or Table for each one, your T/D will come back from the Dean's 
office with a note for you to remove these!!\\
All other tables, such as for the NOMENCLATURE (NOM) are updated 
the same way. However, to get a symbol to be added to the NOM, 
you must add a Field before the symbol. For example, suppose 
you wanted the symbol 
\ensuremath{\beta}\tab  Angle of Attackom\ensuremath{\beta} to be added to the NOM. You would have to move your 
cursor immediately before the symbol (between the l and the beta 
for this example), go to the Insert menu, and click Field. Then, 
in the Field Names box, scroll down to TC and select TC, select 
Options, type in the text window in quotes the description for 
your symbol, and follow that with a /f NOM. Thus, \ensuremath{\beta} will 
be defined as follows:


\tab TC ``Angle of Attack'' /f NOM.


Now, you should add the symbol to this TC entry. To accomplish 
this, you need to have a Show/Hide Paragraph Code button (refer 
to your documentation) so you can click it to show all codes. 
Then, once the codes are shown, you will see the text description 
you added, and you should place your cursor just inside your 
opening quotes. Next, just add your symbol followed by a tab 
to get the spacing half right. To get the spacing fully correct, 
go to the table, click between your symbol and the description, 
and press the tab button again to insert another tab. Now you'll 
see that it lines up correctly.


To show another example, show the codes on this 
\ensuremath{\alpha}\tab Change in Coupler AngleOM\ensuremath{\alpha}.\\


\subsubsection*{1.1.3.\tab }


\subsubsection*{
Formatting Figures. Figures should be formatted as one below with }

plenty of space above and below the actual Figure (three 1.5 
space carriage returns, or, to be more accurate, four single 
spaces). The title should be below the Figure and be designated 
with Figure x.y. where x is the main Section number it is in 
and y is the number of the Figure in that Section. For example, 
first Section could have Figures 1.1, 1.2, and 1.3. Then, Section 
2 could have 2.1, 2.2, 2.3, and 2.4. Refer to Figures like this: Figure 
1.1 illustrates a classic four-bar linkage where R$_{1}$ is the ground 
link.



All figures and tables must be mentioned before they appear.


\[R_{1} \cos \theta _{1} +R_{2} \cos \theta _{2} +R_{3} \cos \theta _{3}
+R_{4} \cos \theta _{4} =0
\]\tab \tab \tab (1)


\[R_{1} \sin \theta _{1} +R_{2} \sin \theta _{2} +R_{3} \sin \theta _{3}
+R_{4} \sin \theta _{4} =0
\]\tab \tab \tab (2)



Equations should be numbered as above throughout the T/D and 
should have one blank line before and after. 




\subsubsection*{1.1.4.\tab }


\subsubsection*{
Formatting Tables. Tables are formatted in much the same way as}

Figures. The title is placed at the top, and 4 blank lines (or 
3 1.5 space lines) are used above and below. And, unlike Figures, 
tables should NOT be used in the middle of a paragraph. They 
should be placed at the end of the paragraph where they are discussed. 
Table 1.1 is one such example.



\begin{center}
Table 1.1. Original STATUS Array for Example Mechanism\\

\end{center}


\begin{tabular}{llllllllllllllllllllllllllll}
\hline
% ROW 1
\multicolumn{1}{|p{0.382in}|}
{\centering
} & 
\multicolumn{1}{p{0.495in}|}
{\multirow{2}{0.495in}{\raggedright Loop \#}} & 
\multicolumn{12}{p{2.776in}|}
{\raggedright
Vector numbers} \\
\cline{1-1}\cline{3-14}\cline{15-15}\cline{17-28}
% ROW 2
\multicolumn{1}{p{0.061in}|}
{\centering
} & 
\multicolumn{1}{p{0.061in}|}
{\multirow{2}{0.061in}{\raggedright }} & 
\multicolumn{12}{p{0.727in}|}
{\centering
1} & 
\multicolumn{1}{|p{0.382in}|}
{\raggedright
2} & 
\multicolumn{1}{p{0.495in}|}
{\raggedright
3} & 
\multicolumn{1}{p{0.218in}|}
{\raggedright
4} & 
\multicolumn{1}{p{0.218in}|}
{\raggedright
5} & 
\multicolumn{1}{p{0.218in}|}
{\raggedright
6} & 
\multicolumn{1}{p{0.218in}|}
{\raggedright
7} & 
\multicolumn{1}{p{0.218in}|}
{\raggedright
8} & 
\multicolumn{1}{p{0.218in}|}
{\raggedright
9} & 
\multicolumn{1}{p{0.218in}|}
{\raggedright
10} & 
\multicolumn{1}{p{0.218in}|}
{\raggedright
11} & 
\multicolumn{1}{p{0.218in}|}
{\raggedright
12} \\
\cline{1-1}\cline{2-2}\cline{3-3}\cline{4-4}\cline{5-5}\cline{6-6}\cline{7-7}\cline{8-8}\cline{9-9}\cline{10-10}\cline{11-11}\cline{12-12}\cline{13-13}\cline{14-14}\cline{15-15}\cline{16-16}\cline{17-17}\cline{18-18}\cline{19-19}\cline{20-20}\cline{21-21}\cline{22-22}\cline{23-23}\cline{24-24}\cline{25-25}\cline{26-26}\cline{27-27}\cline{28-28}
% ROW 3
\multicolumn{1}{p{0.303in}|}
{\raggedright
Mag} & 
\multicolumn{1}{p{0.254in}|}
{\centering
1} & 
\multicolumn{1}{p{0.256in}|}
{\raggedright
0} & 
\multicolumn{1}{p{0.061in}|}
{\raggedright
0} & 
\multicolumn{1}{p{0.061in}|}
{\raggedright
0} & 
\multicolumn{1}{p{0.061in}|}
{\raggedright
0} & 
\multicolumn{1}{p{0.061in}|}
{\raggedright
9} & 
\multicolumn{1}{p{0.061in}|}
{\raggedright
9} & 
\multicolumn{1}{p{0.061in}|}
{\raggedright
9} & 
\multicolumn{1}{p{0.061in}|}
{\raggedright
9} & 
\multicolumn{1}{p{0.061in}|}
{\raggedright
9} & 
\multicolumn{1}{p{0.061in}|}
{\raggedright
9} & 
\multicolumn{1}{p{0.061in}|}
{\raggedright
9} & 
\multicolumn{1}{p{0.061in}|}
{\raggedright
9} \\
\cline{1-1}\cline{2-2}\cline{3-3}\cline{4-4}\cline{5-5}\cline{6-6}\cline{7-7}\cline{8-8}\cline{9-9}\cline{10-10}\cline{11-11}\cline{12-12}\cline{13-13}\cline{14-14}\cline{15-15}\cline{16-16}\cline{17-17}\cline{18-18}\cline{19-19}\cline{20-20}\cline{21-21}\cline{22-22}\cline{23-23}\cline{24-24}\cline{25-25}\cline{26-26}\cline{27-27}\cline{28-28}
% ROW 4
\multicolumn{1}{p{0.061in}|}
{\raggedright
Dir} & 
\multicolumn{1}{p{0.061in}|}
{\centering
1} & 
\multicolumn{1}{p{0.061in}|}
{\raggedright
1} & 
\multicolumn{1}{|p{0.382in}|}
{\raggedright
1} & 
\multicolumn{1}{p{0.495in}|}
{\raggedright
1} & 
\multicolumn{1}{p{0.218in}|}
{\raggedright
0} & 
\multicolumn{1}{p{0.218in}|}
{\raggedright
9} & 
\multicolumn{1}{p{0.218in}|}
{\raggedright
9} & 
\multicolumn{1}{p{0.218in}|}
{\raggedright
9} & 
\multicolumn{1}{p{0.218in}|}
{\raggedright
9} & 
\multicolumn{1}{p{0.218in}|}
{\raggedright
9} & 
\multicolumn{1}{p{0.218in}|}
{\raggedright
9} & 
\multicolumn{1}{p{0.218in}|}
{\raggedright
9} & 
\multicolumn{1}{p{0.218in}|}
{\raggedright
9} \\
\cline{1-1}\cline{2-2}\cline{3-3}\cline{4-4}\cline{5-5}\cline{6-6}\cline{7-7}\cline{8-8}\cline{9-9}\cline{10-10}\cline{11-11}\cline{12-12}\cline{13-13}\cline{14-14}\cline{15-15}\cline{16-16}\cline{17-17}\cline{18-18}\cline{19-19}\cline{20-20}\cline{21-21}\cline{22-22}\cline{23-23}\cline{24-24}\cline{25-25}\cline{26-26}\cline{27-27}\cline{28-28}
% ROW 5
\multicolumn{1}{p{0.303in}|}
{\raggedright
Mag} & 
\multicolumn{1}{p{0.254in}|}
{\centering
2} & 
\multicolumn{1}{p{0.256in}|}
{\raggedright
9} & 
\multicolumn{1}{p{0.061in}|}
{\raggedright
9} & 
\multicolumn{1}{p{0.061in}|}
{\raggedright
9} & 
\multicolumn{1}{p{0.061in}|}
{\raggedright
9} & 
\multicolumn{1}{p{0.061in}|}
{\raggedright
0} & 
\multicolumn{1}{p{0.061in}|}
{\raggedright
0} & 
\multicolumn{1}{p{0.061in}|}
{\raggedright
1} & 
\multicolumn{1}{p{0.061in}|}
{\raggedright
0} & 
\multicolumn{1}{p{0.061in}|}
{\raggedright
9} & 
\multicolumn{1}{p{0.061in}|}
{\raggedright
9} & 
\multicolumn{1}{p{0.061in}|}
{\raggedright
9} & 
\multicolumn{1}{p{0.061in}|}
{\raggedright
9} \\
\cline{1-1}\cline{2-2}\cline{3-3}\cline{4-4}\cline{5-5}\cline{6-6}\cline{7-7}\cline{8-8}\cline{9-9}\cline{10-10}\cline{11-11}\cline{12-12}\cline{13-13}\cline{14-14}\cline{15-15}\cline{16-16}\cline{17-17}\cline{18-18}\cline{19-19}\cline{20-20}\cline{21-21}\cline{22-22}\cline{23-23}\cline{24-24}\cline{25-25}\cline{26-26}\cline{27-27}\cline{28-28}
% ROW 6
\multicolumn{1}{p{0.061in}|}
{\raggedright
Dir} & 
\multicolumn{1}{p{0.061in}|}
{\centering
2} & 
\multicolumn{1}{p{0.061in}|}
{\raggedright
9} & 
\multicolumn{1}{|p{0.382in}|}
{\raggedright
9} & 
\multicolumn{1}{p{0.495in}|}
{\raggedright
9} & 
\multicolumn{1}{p{0.218in}|}
{\raggedright
9} & 
\multicolumn{1}{p{0.218in}|}
{\raggedright
1} & 
\multicolumn{1}{p{0.218in}|}
{\raggedright
1} & 
\multicolumn{1}{p{0.218in}|}
{\raggedright
0} & 
\multicolumn{1}{p{0.218in}|}
{\raggedright
0} & 
\multicolumn{1}{p{0.218in}|}
{\raggedright
9} & 
\multicolumn{1}{p{0.218in}|}
{\raggedright
9} & 
\multicolumn{1}{p{0.218in}|}
{\raggedright
9} & 
\multicolumn{1}{p{0.218in}|}
{\raggedright
9} \\
\cline{1-1}\cline{2-2}\cline{3-3}\cline{4-4}\cline{5-5}\cline{6-6}\cline{7-7}\cline{8-8}\cline{9-9}\cline{10-10}\cline{11-11}\cline{12-12}\cline{13-13}\cline{14-14}\cline{15-15}\cline{16-16}\cline{17-17}\cline{18-18}\cline{19-19}\cline{20-20}\cline{21-21}\cline{22-22}\cline{23-23}\cline{24-24}\cline{25-25}\cline{26-26}\cline{27-27}\cline{28-28}
% ROW 7
\multicolumn{1}{p{0.303in}|}
{\raggedright
Mag} & 
\multicolumn{1}{p{0.254in}|}
{\centering
3} & 
\multicolumn{1}{p{0.256in}|}
{\raggedright
0} & 
\multicolumn{1}{p{0.061in}|}
{\raggedright
9} & 
\multicolumn{1}{p{0.061in}|}
{\raggedright
9} & 
\multicolumn{1}{p{0.061in}|}
{\raggedright
9} & 
\multicolumn{1}{p{0.061in}|}
{\raggedright
9} & 
\multicolumn{1}{p{0.061in}|}
{\raggedright
9} & 
\multicolumn{1}{p{0.061in}|}
{\raggedright
9} & 
\multicolumn{1}{p{0.061in}|}
{\raggedright
9} & 
\multicolumn{1}{p{0.061in}|}
{\raggedright
0} & 
\multicolumn{1}{p{0.061in}|}
{\raggedright
0} & 
\multicolumn{1}{p{0.061in}|}
{\raggedright
1} & 
\multicolumn{1}{p{0.061in}|}
{\raggedright
0} \\
\hline
\end{tabular}




\subsection*{1.2.\tab }


\subsection*{
MORE EQUATIONS AND FIGURES}

Here are some more equations and Figures for illustration. Notice 
how the Figure number includes Section 1 again and increments 
from that already given above.


\[\theta _{3} =2\tan ^{-1} \left( \frac{-E\pm \sqrt{E^{2} -4DF} }{2D}
\right) 
\]\tab \tab \tab \tab (3)


\[\theta _{4} =2\tan ^{-1} \left( \frac{-B\pm \sqrt{B^{2} -4AC} }{2A}
\right) 
\]\tab \tab \tab \tab (4)



These are new example equations. Note that they are numbers 3 
and 4. In Section 2 you will see some more that are 5 and 6.\\
Shown below is the third figure.






\section*{2.\tab }


\section*{MISCELLANEOUS INFORMATION}
\section*{}


\subsection*{2.1.\tab }


\subsection*{
THIS IS SUBSECTION LEVEL ONE}

You're one Section closer to being finished. All that needs to 
be covered now is that this is the way to reference material 
from outside sources [1].


This sentence is used to take up space, but touches on something 
noted in a footnote.\footnote{Here's the text of the footnote.} The 
footnote is at the bottom of the page.




\subsection*{2.2.\tab }


\subsection*{
THIS IS SUBSECTION LEVEL TWO}

This is a sentence to take up space. Below is another Figure 
for a different Section. Notice that it is Figure 2.1 reflecting 
the first Figure in Section 2.





\begin{center}
Figure 2.1. Vector Loop Decomposition of placeLoop 1


\end{center}


This is the next line after Figure 2.1.\\


\subsubsection*{2.2.1.\tab }


\subsubsection*{
Second Subsection. This is the second subsection of Section 2. 
Note the}

difference between this and the others.\\


\subsubsection*{2.2.2.\tab }


\subsubsection*{
Second Subsection. This is the second subsection of Section 2. 
Note the}

difference between this and the others.\\
2.2.2.1\tab 
Third subsection. This is the third subsection of Section 2. 
Note the


diference between this and the others. And further subsections 
would continue as these do, but there are not styles beyond the 
next level.\\
2.2.2.2\tab 
Third subsection. This is the third subsection of Section 2. 
Note the


diference between this and the others. And further subSections 
would continue as these do, but there are not styles beyond the 
next level.\\
2.2.2.2.1\tab 
Fourth subsection. This is the last subsection that has a style 
sheet 


applied. The heading is underlined in this case. This level of 
subsection should be avoided at all cost. One can instead use 
bullets. To add more subSection heading formats, use the Style 
option under the Format Menu. Note that the subsection title 
is in sentence style.\\
2.2.2.2.2\tab 
Fourth subsection will illustrate the case with very long title. 
This is 


the last subsection that has a style sheet applied. To add more 
subsection heading formats, use the Style option under the Format 
Menu.






\section*{3.\tab }


\section*{THIS IS SECTION THREE}
\section*{}


\subsection*{3.1.\tab }


\subsection*{
THIS IS SUBSECTION LEVEL ONE}

This is a sentence to take up space. This is a sentence to take 
up space. This is a sentence to take up space. This is a sentence 
to take up space. This is a sentence to take up space. This is 
a sentence to take up space. This is a sentence to take up space.\\


\subsubsection*{3.1.1.\tab }


\subsubsection*{
Second Subsection. Note the first line continues on the same}

line as the the Section title. This is a sentence to take up 
space. This is a sentence to take up space. This is a sentence 
to take up space. This is a sentence to take up space.\\


\subsubsection*{3.1.2.\tab }


\subsubsection*{
Second Subsection. Note the first line continues on the same}

line as the Section title. This is a sentence to take up space. 
This is a sentence to take up space. This is a sentence to take 
up space. This is a sentence to take up space.\\
3.1.2.1\tab 
Third subsection. Note the first line continues on the


sme as the Section title. This is a sentence to take up space. 
This is a sentence to takesup space. This is a sentence to take 
up space. This is a sentence to take up space.\\
3.1.2.2\tab 
Third subsection. Note the first line continues on the


sme as the Section title. This is a sentence to take up space. 
This is a sentence to takesup space. This is a sentence to take 
up space. This is a sentence to take up space.\\
3.1.2.2.1\tab 
Fourth subsection. This is the last subsection that has a style 
sheet 


applied. To add more subSection heading formats, use the Style 
option under the Format Menu.\\
3.1.2.2.2\tab 
Fourth subsection. This is the last subsection that has a style 
sheet 


applied. To add more subSection heading formats, use the Style 
option under the Format Menu.


\newpage

Double click on the rectangle placed over the page number and 
change the Line color to ``No Line''. Now you can place the rectangle 
such that it hides the page number.


APPENDIX A.\\
THIS IS AN APPENDIX


\newpage
Put your appendix information here. Note that the format used 
here is for a T/D that has more than one appendix. The format 
for this is each appendix must have its own title sheet with 
the word APPENDIX, in uppercase letters, followed by an uppercase 
letter and centered on the page. You should then have two blank 
lines (or one 1.5 space line) and then the TITLE of the appendix 
in uppercase letters. And, you do NOT include a page number on 
your hardcopy, however it is counted in the overall page count. 
Finally, it should be listed in the TOC.\\
For a T/D with a single appendix, the word APPENDIX must be centered 
at the top of the page and the material should start on the same 
page. Note that there is NOT a letter distinction for this type.\\
Also note that these will have to be doctored in the TOC to match 
what it looks like there.


PAGE NUMBER SHOULD NOT BE PRESENT ON APPENDIX COVER PAGES. ONE 
SOLUTION IS TO HAVE THE APPENDICES IN A DIFFERENT FILE.  ANOTHER 
EASY WAY TO DO IT IS TO DRAW A RECTANGLE WITHOUT BORDER OVER THE 
PAGE NUMBER AS SHOWN IN THE PREVIOUS PAGE.

\newpage

\begin{center}
APPENDIX B.\\
THIS IS ANOTHER APPENDIX


\end{center}

\newpage
This is where you put the info for the second appendix. This 
is a sentence to take up space. This is a sentence to take up 
space. This is a sentence to take up space. This is a sentence 
to take up space. This is a sentence to take up space. This is 
a sentence to take up space. This is a sentence to take up space. 
This is a sentence to take up space. This is a sentence to take 
up space. This is a sentence to take up space. This is a sentence 
to take up space. 


READ THIS

You can shrink text to print programs (i.e., condense print) 
or long explanations. The margins must be maintained as in the 
rest of the document and figures and tables should stand out clearly.

\newpage

\begin{center}
BIBLIOGRAPHY


\end{center}


NOTE:  For references, comma comes before closing quotation mark.

[1]\tab \tab J. McCardle and D. Chester, ``Measuring an Asynchronous Processor's 
Power  \\
and Noise,'' Synopsys User Group Conference (SNUG), CityplaceBoston, 
2001.



[2]\tab \tab \tab C. L. Seitz, ``System Timing,'' in Introduction to VLSI Systems, 
Addison-Wesley, pp. 218-262, 1980.

[3]\tab \tab \tab http://www.sce.carleton.ca/faculty/chinneck/thesis.html. Organizing Your Thesis, June 2004 (date mentioned here is the 
date on which the website was last visited).


One can follow the numbered listing shown above OR use the alphabetical 
listing shown below. Refer to a writing handbook for the correct 
formats of references. 


ACI Committee 440, Guide for the Design and Construction of Concrete 
Reinforced with \\
\tab FRP Bars (440.1R-01), American Concrete Institute, CityplaceFarmington 
Hills, StateMichigan, \\
\tab 2001, 41 pp.

Canny J, A Computational Approach to Edge Detection. IEEE Transactions 
PAMI 1986, \\
\tab Vol 10, pp. 679-698.


\newpage

\begin{center}
VITA


\end{center}

Provide information about yourself here. You must include your 
full name, Joseph Edward Miner, and your date of birth January 
32, 1891. All degrees earned and year earned need to be included 
in the Vita. Also include the degree and date of the current 
degree earned. This is a sentence to take up space. This is a 
sentence to take up space. This is a sentence to take up space.\\
This is a sentence to take up space. This is a sentence to take 
up space. This is a sentence to take up space. This is a sentence 
to take up space. This is a sentence to take up space. This is 
a sentence to take up space. This is a sentence to take up space. 
This is a sentence to take up space.\\
This is a sentence to take up space. This is a sentence to take 
up space. This is a sentence to take up space. This is a sentence 
to take up space. This is a sentence to take up space.\\
Finally, don't forget to include a blank unnumbered page at the 
end for your hardcopy.




\end{document}
